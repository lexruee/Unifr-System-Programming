\documentclass[12pt]{article}
\usepackage{url,graphicx,tabularx,array,geometry}
\usepackage{listings}
\usepackage[utf8]{inputenc}
\usepackage{setspace}
\setlength{\parskip}{1ex} %--skip lines between paragraphs
\setlength{\parindent}{0pt} %--don't indent paragraphs

%-- Commands for header
\renewcommand{\title}[1]{\textbf{#1}\\}
\renewcommand{\line}{\begin{tabularx}{\textwidth}{X>{\raggedleft}X}\hline\\\end{tabularx}\\[-0.5cm]}
\newcommand{\leftright}[2]{\begin{tabularx}{\textwidth}{X>{\raggedleft}X}#1%
& #2\\\end{tabularx}\\[-0.5cm]}

\onehalfspacing
%\linespread{2} %-- Uncomment for Double Space
\begin{document}

\title{Imperative and System Programming Autumn 2013}
\line
\leftright{\today}{Alexander Rüedlinger, 08-129-710} %-- left and right positions in the header
\section*{Series 2}
\subsection*{Part 1: C}
\subsubsection*{1. Type Conversion, Casting and ASCII Table}


\subsubsection*{2. Constant, Variable, Escape Character '\textbackslash', and Octal resp. Hexadecimal Digits}

\subsubsection*{3. enum Type}

\subsubsection*{4. Logical Expressions}

\subsubsection*{5. Conditional Expression}

\subsubsection*{6. Bitwise operator}

\subsubsection*{7. Order of Evaluation}
\paragraph{a)}
In the C programming language it isn't specified in which order the operands of an operator are evaluated. The only exceptions are the following operators: and, or, tenary and the comma operator

\paragraph{b)} 

\subsection*{Part 2: System and Unix}
\subsubsection*{8. Memory Diagrams}

\end{document}
