\documentclass[12pt]{article}
\usepackage{url,graphicx,tabularx,array,geometry}
\usepackage{listings}
\usepackage[utf8]{inputenc}
\usepackage{setspace}
\usepackage{amsmath}
\setlength{\parskip}{1ex} %--skip lines between paragraphs
\setlength{\parindent}{0pt} %--don't indent paragraphs

%-- Commands for header
\renewcommand{\title}[1]{\textbf{#1}\\}
\renewcommand{\line}{\begin{tabularx}{\textwidth}{X>{\raggedleft}X}\hline\\\end{tabularx}\\[-0.5cm]}
\newcommand{\leftright}[2]{\begin{tabularx}{\textwidth}{X>{\raggedleft}X}#1%
& #2\\\end{tabularx}\\[-0.5cm]}

\onehalfspacing
%\linespread{2} %-- Uncomment for Double Space
\begin{document}

\title{Imperative and System Programming Autumn 2013}
\line
\leftright{\today}{Alexander Rüedlinger, 08-129-710} %-- left and right positions in the header
\section*{Series 2}
\subsection*{Part 1: C}
\subsubsection*{1. Type Conversion, Casting and ASCII Table}
Remark: In the ASCII table the integer value 65 corresponds to the character 'A'.
\begin{lstlisting}
(1) printf("%c %i\n", c, c);
\end{lstlisting}

In the C language char datatypes are implicit casted to the datatype int. Therefore an explicit cast is not necessary in the previous example.
The statement (1) will print:\\
A 65

\begin{lstlisting}
(2) printf("%c %i\n", i, i);
\end{lstlisting}
In the statement (2) the first format specifier \%c converts the int value i to a char 'A'. The second format specifier just replaces the value in the string.
So the output will be:\\
A 65

\begin{lstlisting}
printf("%f %i\n", pi, (int)pi);
\end{lstlisting}
The first format specifier takes a variable of type float and the second one a variable of type int.
Because the third argument in printf() is casted to an int, the statement above will print:\\

3.14 3

\subsubsection*{2. Constant, Variable, Escape Character '\textbackslash', and Octal resp. Hexadecimal Digits}
\begin{lstlisting}
(0) #define AT '\100'
\end{lstlisting}
The line above creates a symbolic constant with the content '\textbackslash 100'.
Every occurrence of the symbolic constant AT will be substituted with '\textbackslash 100' by the pre-processor.

\begin{lstlisting}
(1) char at = '\x40'; // variable in hexadecimal
\end{lstlisting}
This line creates a char variable using an hexadecimal constant. The hexadecimal value is equals to 64 in the decimal number system.
\begin{equation}
\text{at} = [4 \: 0]_{16} = [0100 \: 0000] = 64
\end{equation} 
Because character values are internal stored as integers, the value 64 corresponds to the @ symbol according the ASCII table.

\begin{lstlisting}
(2) printf("%c %i %o %x\n", '@', '@', '@', '@');
\end{lstlisting}
The printf statement uses four different format specifiers for the character @.
The \%c specifier will be substituted with the character @, \%i will be replaced by the integer value 64. The other specifiers \%o and \%x will convert to the integer value 64 to octal and hex. 
\begin{equation}
\begin{split}
{[64]}_{10} = {[2^6]}_{10} = {[0100 \: 0000]}_2 \\
= [4 \: 0]_{8} =[4 \: 0]_{16}
\end{split}
\end{equation}

So the output of this printf statement will be:

\begin{lstlisting}
@ 64 
\end{lstlisting}


\begin{lstlisting}
(3) printf("%c %i %o %x\n", AT, AT, AT, AT);
\end{lstlisting}


\begin{lstlisting}
(4) printf("%c %i %o %x\n", at, at, at, at);
\end{lstlisting}


\subsubsection*{3. enum Type}

\subsubsection*{4. Logical Expressions}

\subsubsection*{5. Conditional Expression}

\subsubsection*{6. Bitwise operator}

\subsubsection*{7. Order of Evaluation}
\paragraph{a)}
In the C programming language it isn't specified in which order the operands of an operator are evaluated. The only exceptions are the following operators: and, or, tenary and the comma operator

\paragraph{b)} 

\subsection*{Part 2: System and Unix}
\subsubsection*{8. Memory Diagrams}

\end{document}
