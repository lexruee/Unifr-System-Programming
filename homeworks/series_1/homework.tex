\documentclass[12pt]{article}
\usepackage{url,graphicx,tabularx,array,geometry}
\usepackage{listings}
\usepackage[utf8]{inputenc}
\usepackage{setspace}
\setlength{\parskip}{1ex} %--skip lines between paragraphs
\setlength{\parindent}{0pt} %--don't indent paragraphs

%-- Commands for header
\renewcommand{\title}[1]{\textbf{#1}\\}
\renewcommand{\line}{\begin{tabularx}{\textwidth}{X>{\raggedleft}X}\hline\\\end{tabularx}\\[-0.5cm]}
\newcommand{\leftright}[2]{\begin{tabularx}{\textwidth}{X>{\raggedleft}X}#1%
& #2\\\end{tabularx}\\[-0.5cm]}

\onehalfspacing
%\linespread{2} %-- Uncomment for Double Space
\begin{document}

\title{Imperative and System Programming Autumn 2013}
\line
\leftright{\today}{Alexander Rüedlinger, 08-129-710} %-- left and right positions in the header
\section*{Series 1}
\subsection*{Exercise 3: Word count}
\subsubsection*{Example 1}
\begin{lstlisting}
% ./wcount
\end{lstlisting}
This command executes the wcount program. It uses the function getchar() from the stdio.h library to read the next character from the standard input (stdin).  
   
Because we haven't specified an input redirection ($<$) the program reads directly the input from the terminal. 
Entering a phrase like hello world and terminating the program with Ctrl-D results in:
\begin{lstlisting}
1 2 12
\end{lstlisting}

\subsubsection*{Example 2}
\begin{lstlisting}
% ./wcount < wcount.c
\end{lstlisting}
In this example an input redirection is used to feed the programm wcount with data.  
It reads the entire file wcount.c and computes mectris such as:
\begin{itemize}
	\item number of line feeds,
	\item bytes (number of chars) 
	\item and words.
\end{itemize}

\subsubsection*{Example 3}
\begin{lstlisting}
% ./wcount < wcount > test
\end{lstlisting}
Here are two input/ouput operators applied in conjunction with ./wcount.  
The operator $<$ reads the binary file wcount, well it reads itself ;-).  
After that it computes the mectrics (bytes, words, line feeds). 
The produced output is redirected using the output redirection $>$ operator to the file test. 


\subsubsection*{Example 4}
\begin{lstlisting}
% cat wcount.c | ./wcount
\end{lstlisting}
This example shows the cat command and the pipe $|$ operator.  
In a first step the command cat prints the source code of wcount.c on the standard output (stdout).  
Because the pipe operator follows the command cat its output is redirected to  wcount.
As a final step wcount computes the metrics and prints them on the stdout.


\subsubsection*{Example 5}
\begin{lstlisting}
% grep { wcount.c
\end{lstlisting}
grep stands for global regular expression. This program searches the input files for lines containing a match to the given pattern.  
  
In this example grep searches for multiple occurrences of the left curly brace in the source code wcount.c.  
grep prints only the matched lines of wcount.c where left curly braces are present.


\subsubsection*{Example 6}
\begin{lstlisting}
% grep { wcount.c | ./wcount
\end{lstlisting}
As explained before grep uses the search pattern $\{$.  
It prints the matched lines on stdout. Due the application of the pipe operator the output from grep is redirected to wcount. Finally wcount computes the mectris of wcount.c and prints them on stdout.

\subsubsection*{Example 7}
\begin{lstlisting}
% grep -l { * | ./wcount
\end{lstlisting}
The option -l supresses the normal output. It prints the name of each input file from which output would normaly have been printed.
The star symbol  $*$ means that all files of the current directory should be used as input files.

The command above searches for the pattern $\{$ in all files of the current directory and prints all file names where a match has been found. This output is redirected by the pipe operator $|$ to the ./wcount. Finally ./wcount computes the metrics of the list of file names.

\subsubsection*{Short example}

\begin{lstlisting}
% grep -l { * 
\end{lstlisting}
The command above produces the following output on my system:

\begin{lstlisting}
homework.log
homework.md
homework.pdf
homework.synctex.gz
homework.tex
wcount
wcount.c
\end{lstlisting}

This list of file names is redirected to ./wcount which prints on stdout:
\begin{lstlisting}
7 7 87
\end{lstlisting}
\begin{itemize}
	\item 7 lines - 7 file names
	\item 7 words
	\item 87 - number of chars
\end{itemize}

\subsection*{Exercise 4: wc Unix Utility}
\begin{lstlisting}
wc [options] [file ...]
\end{lstlisting}
\subsubsection*{Options}
With the possibility to specify options like -c, -m or -l we're able to customize the output of the program wc. 
So if we're only interesed in the number of characters we can use the -c option.
Our wcount program has not this flexibilty such that its ouput is fixed.

\subsubsection*{File option / Input files}
Our program wcount doesn't support reading input files. In the examples above we were always forced to use the input operator $<$ in order to read a file.


\end{document}
