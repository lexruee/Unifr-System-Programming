\documentclass[12pt]{article}
\usepackage{url,graphicx,tabularx,array,geometry}
\usepackage{listings}
\usepackage[utf8]{inputenc}
\usepackage{setspace}
\usepackage{amsmath}
\usepackage{enumitem}
\setlength{\parskip}{1ex} %--skip lines between paragraphs
\setlength{\parindent}{0pt} %--don't indent paragraphs

%-- Commands for header
\renewcommand{\title}[1]{\textbf{#1}\\}
\renewcommand{\line}{\begin{tabularx}{\textwidth}{X>{\raggedleft}X}\hline\\\end{tabularx}\\[-0.5cm]}
\newcommand{\leftright}[2]{\begin{tabularx}{\textwidth}{X>{\raggedleft}X}#1%
& #2\\\end{tabularx}\\[-0.5cm]}

\onehalfspacing
%\linespread{2} %-- Uncomment for Double Space
\begin{document}

\title{Imperative and System Programming Autumn 2013}
\line
\leftright{\today}{Alexander Rüedlinger, 08-129-710, Group 05} %-- left and right positions in the header
\section*{Series 5}
\subsection*{3. Caclulator}
\paragraph{a)} 

\begin{lstlisting}
int getch(void)	// get a (possibly pushed back) character --- kr79
{
    return (bufp > 0) ? buf[--bufp] : getchar();
}
\end{lstlisting}

\subparagraph{No system call...} If the variable bufp is greater than zero then there is a at least one character in the buffer buf[].

If bufp $>$ 0 then the function getch() doesn't perform an IO system function call by calling getchar(). It returns instead the last inserted character from the buffer.

If busp $<=$ 0 then the function getch() performs an IO system function call getchar() and reada the next character from the stdin.

\subparagraph{cache memory}
\begin{quote}
[...]a cache is a component that transparently stores data so that future requests for that data can be served faster. [...]
\end{quote}
Source: wikipedia.com, keyword: cache

In this example the buffer acts as a temporary storage of a character that has been eaten by getop() but which is needed in a further processing step.

The buffer can be regarded as a cache memory because it allows to access a pre-fetched character from the stdin in a fast way. Compared to a system call an access to a cache memory to read a character is much faster.

\paragraph{b)}
To solve this exercise in an easy way I defined the function calls getch() and ungetch() as macros. Here I used the do {}while(0) trick to implement the ungetch function. The function getch() is just replaced by the ternary if-else construct. This way the code must be slightly adjusted:

\begin{lstlisting}
int getop(char s[])
{
    static char buf[BUFSIZE];
    static int bufp = 0;
    ...
\end{lstlisting}

Fore more details see the implementation.

Both implementations the original one and the refactored one produce the same output:
\begin{lstlisting}
./calc < calc_input.txt 
	6
	5
	30
\end{lstlisting}

\subsection*{4. FIFO}
For this exercise I implemented a help function called fifo\_shift() and fifo\_print().

\begin{lstlisting}
void fifo_shift(){
   int i; 
   for(i=1; i < CAPACITY; i++){
     *(base + (i-1)) = *(base + i);
   }
   *(base + (CAPACITY-1)) = 0; //set last position to zero
}
\end{lstlisting} 

\begin{lstlisting}
void fifo_print(){
    int i;
    printf("["); 
    for(i=0; i < CAPACITY; i++){
        printf(" %i ",base[i]);  
    }
    printf("]\n");       
}
\end{lstlisting}

If an overflow happens in the function fifo\_put() then each position in the array base is shifted one position to the left (from i to i-1) and the value passed to fifo\_put() is set at the last position of the base array.

\end{document}