\documentclass[12pt]{article}
\usepackage{url,graphicx,tabularx,array,geometry}
\usepackage{listings}
\usepackage[utf8]{inputenc}
\usepackage{setspace}
\usepackage{amsmath}
\usepackage{enumitem}
\setlength{\parskip}{1ex} %--skip lines between paragraphs
\setlength{\parindent}{0pt} %--don't indent paragraphs

%-- Commands for header
\renewcommand{\title}[1]{\textbf{#1}\\}
\renewcommand{\line}{\begin{tabularx}{\textwidth}{X>{\raggedleft}X}\hline\\\end{tabularx}\\[-0.5cm]}
\newcommand{\leftright}[2]{\begin{tabularx}{\textwidth}{X>{\raggedleft}X}#1%
& #2\\\end{tabularx}\\[-0.5cm]}

\onehalfspacing
%\linespread{2} %-- Uncomment for Double Space
\begin{document}

\title{Imperative and System Programming Autumn 2013}
\line
\leftright{\today}{Alexander Rüedlinger, 08-129-710, Group 05} %-- left and right positions in the header
\section*{Series 3}

\subsection*{1. Loops: while and for}
From a theoretical point of view this a very interesting exercise.  That's because that while and goto programs are both equivalent to turing machines and therefore turing complete. In this sense the C programming language is also turing complete.
The main result in this exercise is of course that each while loop can be rewritten using a goto statement, and each for loop using a while loop. Besides that a loop can be rewritten using recursion.

\paragraph*{a)}\quad
\begin{lstlisting}
#include<stdio.h>
main(){
	int low,high=10;	
	for(low=0; low <= high; low++)
		printf("%i\n",low);
}
\end{lstlisting}
\paragraph*{b)}
\quad
\begin{lstlisting}
#include<stdio.h>

main(){
	int n = 10;
	int low = 0, high = n - 1;
	do {
		printf("%i\n",low++);
	} while(low <= high);
}
\end{lstlisting}

\subsection*{2. goto and labels}
\paragraph*{a)}
\quad
\begin{lstlisting}
#include<stdio.h>
main(){
	int n = 10, i = 0;	
	//enter goto loop
	rep:
	if(i < n){	
		printf("%i\n",i++);		
		goto rep; //repeat
	}
}
\end{lstlisting}
\paragraph*{b)} \quad
\begin{lstlisting}
#include<stdio.h>
/** this is a kickass state machine! */
main(int argc, char* argv[]){
	int i;
	if(argc==2 && (*argv[1]>='0' && *argv[1]<='9'))
		i = *argv[1] - '0';
	else goto end; 

	//jump/state table
	if(i==1) goto case_1;
	else if(i==2) goto case_2;
	else goto _default;	

	//cases
	case_1: printf("first case\n"); goto end;
	case_2: printf("second case\n"); goto end;
	_default: printf("other case\n"); goto end;
	end: ;  //do nothing
}
\end{lstlisting}

\paragraph*{c)}
\quad
\begin{lstlisting}
#include<stdio.h>

void f1(){printf("f1\n");}
void f2(){printf("f2\n");}

main(){
	int cond1 = 0, cond2 = 1, n=3, i;
	_continue:
	for(i = 0; i < n; i++){
		f1();
		if(cond1) goto _break;
		f2();
		cond1 = 1;		
		if(cond2) goto _continue;
	}
	
	_break: ;
}
\end{lstlisting}

\subsection*{UNIT Testing: first contact}

\subsubsection*{a) Testing 1a) anf 1b)}
For these programs I've created a simple text file named n.txt.
It just contains the value 3.

The programs in 1a) and 1b) are tested as follows:
\begin{lstlisting}
 ./Sol1a < n.txt 
> n = low_initial = 0
0
1
2
low_final = 3
\end{lstlisting}

\begin{lstlisting}
./Sol1b < n.txt 
> n = low_initial = 0
0
1
2
low_final = 3
\end{lstlisting}

So both programs produce the same output as Fig1a. Therefore they behave correctly:
\begin{lstlisting}
./Fig1a < n.txt 
> n = low_initial = 0
0
1
2
low_final = 3
\end{lstlisting}

\subsubsection*{b) Testing 2a) and 2b)}
Here I use the same testing strategy. Again a use a file called n.txt which contains a value of 3.

I ran both programs using the pipe operator to read a input file.
\begin{lstlisting}
./Sol2a < n.txt
./Sol2b < n.txt
\end{lstlisting}

For Sol2a I get the output:
\begin{lstlisting}
./Sol2a < n.txt 
> n = 0
1
2
i_final = 3
\end{lstlisting}

For Sol2b I get the output:
\begin{lstlisting}
./Sol2b < n.txt 
> i = other case
\end{lstlisting}

These ouputs are in line with Fig2a and Fig2b.

\subsection*{Binsearch: test and complexity}
\subsubsection*{a)}
To test the binary search algorithm I created a file named numbers.txt with the content below:
\begin{lstlisting}
10
3 0 1
8 1 1
9 2 1
13 3 1
15 4 1
67 5 1
77 6 1
91 7 1
6 8 0
11 9 0
\end{lstlisting}
The first number on the first line is the size of the list that is used in the test. The next lines are encoded as follows. The first number represents the number in the list. The second number specifies the position of the first number in the list. The last number specifies if the test should pass (1) or fail (0).

For example the line 8 1 1 says that the number 8 is on the position 1 in the list and that the test should pass. The line 11 9 0 says that the number 11 is on the position 9 and that the test should fail.

So the test scenario is as follows. A list with values $\begin{bmatrix}
3 & 8 & 9 & 13 & 15 & 67 & 77 & 91 & 6 & 11
\end{bmatrix} $ is passed to the binary search algorithm.

All the values are increasing from left to right. But the last two values are incorrectly positioned.

So the tests that uses values from 3 to 91 should all pass and last two tests which use the values 6 and 11 should fail.

The output of the test programs yields:
\begin{lstlisting}
test:
find num: 3, res pos: 0, expected pos: 0, test ok?: pass
find num: 8, res pos: 1, expected pos: 1, test ok?: pass
find num: 9, res pos: 2, expected pos: 2, test ok?: pass
find num: 13, res pos: 3, expected pos: 3, test ok?: pass
find num: 15, res pos: 4, expected pos: 4, test ok?: pass
find num: 67, res pos: 5, expected pos: 5, test ok?: pass
find num: 77, res pos: 6, expected pos: 6, test ok?: pass
find num: 91, res pos: 7, expected pos: 7, test ok?: pass
find num: 6, res pos: -1, expected pos: 8, test ok?: pass
find num: 11, res pos: -1, expected pos: 9, test ok?: pass
\end{lstlisting} 

As we can see each tests passes. The last two tests are correct because we expected them to fail.

The test program test\_binsearch.c looks like this:
\begin{lstlisting}
/**
 * @author Alexander Rueedlinger
 * @date 2013
 */

#include "kr058.c"
#include<stdio.h>

main(){
	int n, p;
	scanf("%i",&n); //size of the list

	int i, numbers[n], positions[n], should[n];	
	
	int d;
	for(i = 0; i < n; i++){
		d = scanf("%i %i %i",&numbers[i],&positions[i],&should[i]);
		if(d < 3){
			while(getchar()!='\n');	
		}
	}

	printf("test:\n");
	int check, num, res_pos, expected_pos;
	for(i=0; i < n; i++){
		num = numbers[i];
		expected_pos = positions[i];
		res_pos = binsearch(num, numbers, n);
		check = (res_pos == expected_pos) == should[i]; 
		printf("find num: %i, res pos: %i, expected pos: %i, test ok?: %s\n",num,res_pos,expected_pos,check ? "pass" : "fail");
	}
}
\end{lstlisting}
\subsubsection*{b)}
\paragraph{Input}
The algorithm takes as an input an array of n integers A and an integer k.

\paragraph{Output}
If the integer k is found in the array A the algorithm returns its position otherwise it returns -1.

\paragraph{Invariant}
All the integers in the array A are sorted.

\paragraph{Analysis}
In a first step binary search compares the integer k with the median of the array A.

If the median is equals to k then we've found the position of k. In this case the complexity is O(1) which is the best case.

If k is below the median binary search shrinks the bounds of the array A.
In this case it halves the array. The same is true if k is above the median.
In these two cases binary search halves the remaining array until it reaches the size 1.

This having procedure produces a log base 2 sequence. Therefore in the average case and worst case the complexity is O(ln(n)).

\newpage
\subsection*{Remote Access Tutorial}
\subsubsection*{Exercises 1,2,3}
\paragraph*{Exercise 1:}

To get a list of available hosts we can write a simple ruby program like this one:
\begin{lstlisting}
#! /usr/bin/ruby -w
#
#@author	Alexander Rueedlinger
#@matrikel	08-129-710
#@date		2013
#

host_prefix = "diufsppc"
host_postfix = ".unifr.ch"

ids = (701..720)

found_hosts = Array.new
ids.each do |id|
	hostname = "#{host_prefix}#{id}#{host_postfix}"
	out = `ping #{hostname} -c 1`
	if out.include? "1 packets transmitted, 1 received"
		found_hosts << hostname
		puts "found host: #{hostname}"	
	end
end
\end{lstlisting}

So here are some machines;
\begin{lstlisting}
 ./find_hosts.rb
found host: diufsppc703.unifr.ch
found host: diufsppc708.unifr.ch
found host: diufsppc710.unifr.ch
found host: diufsppc712.unifr.ch
found host: diufsppc713.unifr.ch
found host: diufsppc714.unifr.ch
found host: diufsppc716.unifr.ch
found host: diufsppc717.unifr.ch
found host: diufsppc719.unifr.ch
found host: diufsppc720.unifr.ch
\end{lstlisting}

I choose the machine diufsppc719.unifr.ch, because the girl in front of me is logged in on this machine. I hope she doesn't mind.

\begin{lstlisting}
lexruee@death-star:~$ ssh Rueedlin@diufsppc719.unifr.ch
Password:

rueedlin@diufsppc719:~$ 
\end{lstlisting}

Entering the date command yields the following output:
\begin{lstlisting}
rueedlin@diufsppc719:~$ date
Thu Oct  3 14:26:06 CEST 2013
\end{lstlisting}

\paragraph{Questions}
\begin{itemize}
\item 1. The first part rueedlin of the prompt is my username and the second part after the @ is the hostname that is diufsppc719.
\item 2. The command line date has been executed on the remote machine diufsppc719.
\item 3. The given date is the local date of the remote machine.
\end{itemize}

In order to find out what's her username I can use the command w:
\begin{lstlisting}
rueedlin@diufsppc719:~$ w
 13:46:55 up 42 min,  2 users,  load average: 0.23, 0.36, 0.29
USER     TTY      FROM              LOGIN@   IDLE   JCPU   PCPU WHAT
scheubca tty7     :0               13:06   42:14   1:05   0.12s gnome-session
rueedlin pts/0    wlan-per2-146-21 13:43    0.00s  0.24s  0.01s w
\end{lstlisting}

The w command lists all the users that are currently logged in on the machine.

So her username is scheubca.

Let's see which programs she is currently using:

\begin{lstlisting}
ps -aux | grep scheubca
scheubca  2084  4.4  5.1 582780 209976 ?       
Sl   13:06   2:03 /usr/lib/firefox/firefox
scheubca  2132  0.2  1.0 116624 43488 ?        
Sl   13:08   0:05 /opt/Adobe/Reader9/Reader/intellinux/bin/acroread 
/tmp/Traktandenliste_9.10.2013.pdf
scheubca  2256  1.5  2.1 207912 86756 ?        
Sl   13:10   0:39 /usr/lib/openoffice/program/soffice.bin -writer -splash-pipe=6
\end{lstlisting}

The command ps reports a snapshot of the current processes. The man page of ps says the following about the options -aux:
\begin{quote}
The POSIX and UNIX standards require
       that "ps -aux" print all processes owned by a user named "x", as well as printing all
       processes that would be selected by the -a option. 
\end{quote}
 
From this output I can see that she's is currently browsing using the firefox webbrowser. Besides that she has opened adobe reader to read a pdf called Traktandenliste\_9.10.2013.pdf. Another program that she launched is openoffice.

Now, let's check out her home directory:
\begin{lstlisting}
cd ../scheubca
ls
rueedlin@diufsppc719:/home/local/UNIFR/scheubca$ ls
C:\nppdf32Log\debuglog.txt  Desktop  Documents  
Downloads  examples.desktop  greiner.odt  
Music  Pictures  Public  Templates  Videos
\end{lstlisting}

Here I can see that she has a odt-file in her home directory called greiner.odt.

The conclusion of this small exercise is that the home directories of the ubuntu users are not protected!
\paragraph*{Exercise 2:}
The following outputs show a sftp session on the machine diuf719.unifr.ch.

sftp login:
\begin{lstlisting}
$sftp Rueedlin@diufsppc719.unifr.ch
Password:
Connected to diufsppc719.unifr.ch.
\end{lstlisting}
List all files:
\begin{lstlisting}
sftp> ls
examples.desktop    
\end{lstlisting}

Show the path to the current directory:
\begin{lstlisting}
sftp> pwd
Remote working directory: /home/local/UNIFR/rueedlin
\end{lstlisting}

Create a folder named Documents:
\begin{lstlisting}
sftp> mkdir Documents
sftp> ls
Documents           examples.desktop
\end{lstlisting}

Upload the file text.txt into folder Documents:
\begin{lstlisting}
sftp> put test.txt Documents/text.txt
Uploading test.txt to /home/local/UNIFR/rueedlin/Documents/text.txt
test.txt                                      100%   12     0.0KB/s   00:00 
\end{lstlisting}

Download the file test.txt and save it on my local machine in the Documents folder:
\begin{lstlisting}
sftp> get Documents/text.txt /home/lexruee/Documents/test_copy.txt
Fetching /home/local/UNIFR/rueedlin/Documents/text.txt to 
/home/lexruee/Documents/test_copy.txt
/home/local/UNIFR/rueedlin/Documents/text.txt 100%   12     0.0KB/s   00:00 
\end{lstlisting}

\paragraph{Exercise 3}
This exercise could be solved much simpler using the wget command on the remote machine.
For example:
\begin{lstlisting}
lexruee@death-star:~$ 
lexruee@death-star:~$ ssh Rueedlin@diufsppc719.unifr.ch
password:

rueedlin@diufsppc719:
rueedlin@diufsppc719:wget http://diuf.unifr.ch/pai/ip/labs/kr/kr1/kr006.c\  
-o helloworld.c
rueedlin@diufsppc719:gcc helloworld.c -o helloworld
rueedlin@diufsppc719:./helloworld
\end{lstlisting}

However,...

First I download the kr006.c using wget on my local machine:
\begin{lstlisting}
wget http://diuf.unifr.ch/pai/ip/labs/kr/kr1/kr006.c
\end{lstlisting}

In a next step I connect to the remote machine via sftp:
\begin{lstlisting}
sftp Rueedlin@diufsppc720.unifr.ch
\end{lstlisting}

Now, I upload the kr00 6.c file:
\begin{lstlisting}
sftp> put kr006.c kr006.c 
Uploading kr006.c to /home/local/UNIFR/rueedlin/kr006.c
kr006.c                                       100%   73     0.1KB/s   00:00 
\end{lstlisting}

After that I rename the file using mv:
\begin{lstlisting}
mv kr006.c helloworld.c
\end{lstlisting}

Finally I compile and execute the program:
\begin{lstlisting}
gcc helloworld.c -o helloworld && ./helloworld 
hello, world
\end{lstlisting}

\subsubsection*{a) ssh, sftp and shttp}
\paragraph{ssh}
The command ssh which stands for Secure Shell and is an application protocol that allows a user to administrate a remote machine from his local host. The underlying protocol which is used to establish a connection between the remote and local machine is the TCP protocol (transmission control protocol).
 
\paragraph{sftp}
SSH File Transfer Protocol (sftp) is an application protocol that allows  user to upload, download, delete, and rename files on a remote host over the ssh protocol. To put it differently, sftp is above ssh in the protocol stack and uses ssh as a service to communicate.

\paragraph{shttp}
Secure Hypertext Transfer Protocol or short shttp is also an application protocol. It's an extension of the http protocol.

So the differences are:
\begin{itemize}
\item ssh: a secure remote terminal to administrate a remote machine.
\item shttp: a secure extension of the http protocol which encrypts the http header fields and the http payload.
\item sftp: a secure alternative to ftp that uses ssh as underlying communication protocol to encrypt file transfers between hosts.
\end{itemize}

\subsubsection*{b)}
Every system that is remotely accessible is vulnerable to attacks. In the example of an internet server with enabled ssh daemon the biggest security problem is that everyone can try to login. Simple brute force programs could attempt to login with different user names and user passwords until they've found a valid user/password combination.

Another security concerns are the user rights associated with a ssh user account. If such a ssh user has the same rights as root then this is a huge security hole. So it's a really bad idea to allow root users to login via ssh! Such a system can be contaminated and abused as a email spam server. Besides that an evil user can spy out the and damage network. 

The best way to secure a system from evil ssh user is to jail ssh user to their home directory. If the user is jailed into his home directory he is not able to access, create, modify or delete files outside of their home dir.
 

\end{document}